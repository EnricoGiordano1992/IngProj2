\documentclass[a4paper,titlepage]{article}
\usepackage{frontespizio}
\usepackage[english]{babel}
\usepackage[utf8]{inputenc}

\usepackage[a4paper, total={6in, 9in}]{geometry}



\begin{document}
\begin{frontespizio}
\Universita{Verona}
\Dipartimento{Informatica}
\Corso[Laurea]{Informatica}
\Titoletto{Software Engineering}
\Titolo{Second project report}

\Candidato[VR363021]{Giovanni Liboni}
\Candidato[VR359169]{Enrico Giordano}
\Candidato[VR359129]{Alberto Marini}
\Candidato[VR359333]{Alessandro Falda}

\Annoaccademico{2013-2014}
\end{frontespizio}

\tableofcontents

\newpage

\part{Introduction}

This project implements an asynchronous system consists in three principal parts:

\begin{enumerate}

\item a station, that controls velocity of cars;

\item some automatic cars, that set their velocity randomly during the ride and set the ideal velocity thanks to the station;

\item some manual cars, that set their velocity randomly during the ride and receive ``break'' message (but they are not obliged to slow down).

\end{enumerate}

During the execution, is istantiated 50 manual cars and 40 automatic cars and change the speed of each car randomly during the runtime. After a simple ride, the cars decide randomly if they will be exit or not.

\newpage

\part{Graphical interface}

The graphical interface is built with \textit{swing} Java library and consists in two JFrame called ``WallGraphic'' and ``DebugInterface''. The first interface (WallGraphic) is a representation of the situation, composed by a station, some automatic cars and some manual cars. The second interface (DebugInterface) is a textual console that show the program flow, cars display and state and station state. This interface has three option:

\begin{figure}[!h]
\centering
\includegraphics[scale=0.3]{screen.png}
\caption{ScreenShot}
\end{figure}

\begin {enumerate}

\item ``pause'', that stops the program flow acquisition;

\item ``resume'', that resume the program flow acquisition (and enable autoscroll);

\item ``watch'', that disable or enable the autoscroll of the scrollbar.    

\end{enumerate}

Only 200 rows are shown in the screen; after 200 rows, DebugInterface clean itself and delete old rows. This was made because after 200 setText in the same JLabel probably will crash the program.

This is composed by a JFrame that contains a JPanel that contains a JLabel with a black background and a text that was updated by station and cars display.

~

WallGraphic is composed by a JFrame that contains a JPanel with a ``wallpaper'' (the circuit). This JPanel contains in different levels (Z ordered) a station (that is a JLabel with an image) and some cars (that are a JLabels with an image). The cars move theirself in asynchronous way into the ride in 10 different direction (in order to approximate the elliptical path).

The car label have a superior JLabel that contains its id and the car type: the ``M'' letter represent the ``Manual car'' and the ``A'' letter represent the ``Automatic car''. When a car change its direction (X orientation), it changes its image.

\begin{figure}[!h]
\centering
\includegraphics[scale=0.2]{../car.png}
\caption{car label}
\end{figure}


\begin{figure}[!h]
\centering
\includegraphics[scale=0.2]{../car2.png}
\caption{car label (different orientation)}
\end{figure}

\begin{figure}[!h]
\centering
\includegraphics[scale=0.5]{../radio.png}
\caption{car label}
\end{figure}


Every car is a graphical thread that execute the move() method in order to move itself during the ride; when a car exit to the circuit, the thread dead. The velocity is graphically represented by a thread sleep during the movement.

~

This is the Z order of the JLabel:

~

~ ~ ~ ~ ~-1: background image;

~

~ ~ ~ ~ ~ 0: station;

~

~ ~ ~ ~ ~-1: cars;

~

In this way, the cars will pass graphically behind the station and on the background image.


The graphical class are contained in the \textit{``graphics''} package.

\section{Design pattern for graphical interface}

Every class of this project must use theese graphical interface, so it was implement a \textbf{Façade pattern}. In fact there is a general class, ``ScenarioGraphic'', that include in itself every graphical istance; in this way, when a class must use a graphical istante, simply call the ScenarioGraphic methods and ScenarioGraphic can modify the graphical istances. 

\begin{figure}[!h]
\centering
\includegraphics[scale=0.5]{facade.png}
\caption{Façade pattern}
\end{figure}



\section{Swing bug}

\begin{verbatim}
Exception in thread "Thread-1" java.lang.NullPointerException
	at javax.swing.BufferStrategyPaintManager.flushAccumulatedRegion(BufferStrategyPaintManager.java:427)
	at javax.swing.BufferStrategyPaintManager.copyArea(BufferStrategyPaintManager.java:351)
	at javax.swing.RepaintManager.copyArea(RepaintManager.java:1232)
	at javax.swing.JViewport.blitDoubleBuffered(JViewport.java:1621)
	at javax.swing.JViewport.windowBlitPaint(JViewport.java:1590)
	at javax.swing.JViewport.setViewPosition(JViewport.java:1135)
	at javax.swing.plaf.basic.BasicScrollPaneUI$Handler.vsbStateChanged(BasicScrollPaneUI.java:1044)
	at javax.swing.plaf.basic.BasicScrollPaneUI$Handler.stateChanged(BasicScrollPaneUI.java:1033)
	at javax.swing.DefaultBoundedRangeModel.fireStateChanged(DefaultBoundedRangeModel.java:365)
	at javax.swing.DefaultBoundedRangeModel.setRangeProperties(DefaultBoundedRangeModel.java:302)
	at javax.swing.DefaultBoundedRangeModel.setValue(DefaultBoundedRangeModel.java:168)
	at javax.swing.JScrollBar.setValue(JScrollBar.java:463)
	at graphics.DebugInterface.run(DebugInterface.java:50)
	at java.lang.Thread.run(Thread.java:701)
\end{verbatim}

\end{document}


